\documentclass[ucs, notheorems, handout]{beamer}
\usetheme[numbers,totalnumbers,compress, nologo]{Statmod}
\usefonttheme[onlymath]{serif}
\setbeamertemplate{navigation symbols}{}

\usepackage{graphicx,subcaption,ragged2e}

%new calligraphic font for subspaces 
\usepackage{euscript}
\newcommand{\cA}{\EuScript{A}}
\newcommand{\cB}{\EuScript{B}}
\newcommand{\cC}{\EuScript{C}}
\newcommand{\cD}{\EuScript{D}}
\newcommand{\cE}{\EuScript{E}}
\newcommand{\cF}{\EuScript{F}}
\newcommand{\cG}{\EuScript{G}}
\newcommand{\cH}{\EuScript{H}}
\newcommand{\cI}{\EuScript{I}}
\newcommand{\cJ}{\EuScript{J}}
\newcommand{\cK}{\EuScript{K}}
\newcommand{\cL}{\EuScript{L}}
\newcommand{\cM}{\EuScript{M}}
\newcommand{\cN}{\EuScript{N}}
\newcommand{\cO}{\EuScript{O}}
\newcommand{\cP}{\EuScript{P}}
\newcommand{\cQ}{\EuScript{Q}}
\newcommand{\cR}{\EuScript{R}}
\newcommand{\cS}{\EuScript{S}}
\newcommand{\cT}{\EuScript{T}}
\newcommand{\cU}{\EuScript{U}}
\newcommand{\cV}{\EuScript{V}}
\newcommand{\cW}{\EuScript{W}}
\newcommand{\cX}{\EuScript{X}}
\newcommand{\cY}{\EuScript{Y}}
\newcommand{\cZ}{\EuScript{Z}}

%font for text indices like transposition X^\mathrm{T}
\newcommand{\rmA}{\mathrm{A}}
\newcommand{\rmB}{\mathrm{B}}
\newcommand{\rmC}{\mathrm{C}}
\newcommand{\rmD}{\mathrm{D}}
\newcommand{\rmE}{\mathrm{E}}
\newcommand{\rmF}{\mathrm{F}}
\newcommand{\rmG}{\mathrm{G}}
\newcommand{\rmH}{\mathrm{H}}
\newcommand{\rmI}{\mathrm{I}}
\newcommand{\rmJ}{\mathrm{J}}
\newcommand{\rmK}{\mathrm{K}}
\newcommand{\rmL}{\mathrm{L}}
\newcommand{\rmM}{\mathrm{M}}
\newcommand{\rmN}{\mathrm{N}}
\newcommand{\rmO}{\mathrm{O}}
\newcommand{\rmP}{\mathrm{P}}
\newcommand{\rmQ}{\mathrm{Q}}
\newcommand{\rmR}{\mathrm{R}}
\newcommand{\rmS}{\mathrm{S}}
\newcommand{\rmT}{\mathrm{T}}
\newcommand{\rmU}{\mathrm{U}}
\newcommand{\rmV}{\mathrm{V}}
\newcommand{\rmW}{\mathrm{W}}
\newcommand{\rmX}{\mathrm{X}}
\newcommand{\rmY}{\mathrm{Y}}
\newcommand{\rmZ}{\mathrm{Z}}

%tt font for time series
\newcommand{\tA}{\mathsf{A}}
\newcommand{\tB}{\mathsf{B}}
\newcommand{\tC}{\mathsf{C}}
\newcommand{\tD}{\mathsf{D}}
\newcommand{\tE}{\mathsf{E}}
\newcommand{\tF}{\mathsf{F}}
\newcommand{\tG}{\mathsf{G}}
\newcommand{\tH}{\mathsf{H}}
\newcommand{\tI}{\mathsf{I}}
\newcommand{\tJ}{\mathsf{J}}
\newcommand{\tK}{\mathsf{K}}
\newcommand{\tL}{\mathsf{L}}
\newcommand{\tM}{\mathsf{M}}
\newcommand{\tN}{\mathsf{N}}
\newcommand{\tO}{\mathsf{O}}
\newcommand{\tP}{\mathsf{P}}
\newcommand{\tQ}{\mathsf{Q}}
\newcommand{\tR}{\mathsf{R}}
\newcommand{\tS}{\mathsf{S}}
\newcommand{\tT}{\mathsf{T}}
\newcommand{\tU}{\mathsf{U}}
\newcommand{\tV}{\mathsf{V}}
\newcommand{\tW}{\mathsf{W}}
\newcommand{\tX}{\mathsf{X}}
\newcommand{\tY}{\mathsf{Y}}
\newcommand{\tZ}{\mathsf{Z}}

%bf font for matrices
\newcommand{\bfA}{\mathbf{A}}
\newcommand{\bfB}{\mathbf{B}}
\newcommand{\bfC}{\mathbf{C}}
\newcommand{\bfD}{\mathbf{D}}
\newcommand{\bfE}{\mathbf{E}}
\newcommand{\bfF}{\mathbf{F}}
\newcommand{\bfG}{\mathbf{G}}
\newcommand{\bfH}{\mathbf{H}}
\newcommand{\bfI}{\mathbf{I}}
\newcommand{\bfJ}{\mathbf{J}}
\newcommand{\bfK}{\mathbf{K}}
\newcommand{\bfL}{\mathbf{L}}
\newcommand{\bfM}{\mathbf{M}}
\newcommand{\bfN}{\mathbf{N}}
\newcommand{\bfO}{\mathbf{O}}
\newcommand{\bfP}{\mathbf{P}}
\newcommand{\bfQ}{\mathbf{Q}}
\newcommand{\bfR}{\mathbf{R}}
\newcommand{\bfS}{\mathbf{S}}
\newcommand{\bfT}{\mathbf{T}}
\newcommand{\bfU}{\mathbf{U}}
\newcommand{\bfV}{\mathbf{V}}
\newcommand{\bfW}{\mathbf{W}}
\newcommand{\bfX}{\mathbf{X}}
\newcommand{\bfY}{\mathbf{Y}}
\newcommand{\bfZ}{\mathbf{Z}}

%bb font for standard spaces and expectation
\newcommand{\bbA}{\mathbb{A}}
\newcommand{\bbB}{\mathbb{B}}
\newcommand{\bbC}{\mathbb{C}}
\newcommand{\bbD}{\mathbb{D}}
\newcommand{\bbE}{\mathbb{E}}
\newcommand{\bbF}{\mathbb{F}}
\newcommand{\bbG}{\mathbb{G}}
\newcommand{\bbH}{\mathbb{H}}
\newcommand{\bbI}{\mathbb{I}}
\newcommand{\bbJ}{\mathbb{J}}
\newcommand{\bbK}{\mathbb{K}}
\newcommand{\bbL}{\mathbb{L}}
\newcommand{\bbM}{\mathbb{M}}
\newcommand{\bbN}{\mathbb{N}}
\newcommand{\bbO}{\mathbb{O}}
\newcommand{\bbP}{\mathbb{P}}
\newcommand{\bbQ}{\mathbb{Q}}
\newcommand{\bbR}{\mathbb{R}}
\newcommand{\bbS}{\mathbb{S}}
\newcommand{\bbT}{\mathbb{T}}
\newcommand{\bbU}{\mathbb{U}}
\newcommand{\bbV}{\mathbb{V}}
\newcommand{\bbW}{\mathbb{W}}
\newcommand{\bbX}{\mathbb{X}}
\newcommand{\bbY}{\mathbb{Y}}
\newcommand{\bbZ}{\mathbb{Z}}

%got font for any case
\newcommand{\gA}{\mathfrak{A}}
\newcommand{\gB}{\mathfrak{B}}
\newcommand{\gC}{\mathfrak{C}}
\newcommand{\gD}{\mathfrak{D}}
\newcommand{\gE}{\mathfrak{E}}
\newcommand{\gF}{\mathfrak{F}}
\newcommand{\gG}{\mathfrak{G}}
\newcommand{\gH}{\mathfrak{H}}
\newcommand{\gI}{\mathfrak{I}}
\newcommand{\gJ}{\mathfrak{J}}
\newcommand{\gK}{\mathfrak{K}}
\newcommand{\gL}{\mathfrak{L}}
\newcommand{\gM}{\mathfrak{M}}
\newcommand{\gN}{\mathfrak{N}}
\newcommand{\gO}{\mathfrak{O}}
\newcommand{\gP}{\mathfrak{P}}
\newcommand{\gQ}{\mathfrak{Q}}
\newcommand{\gR}{\mathfrak{R}}
\newcommand{\gS}{\mathfrak{S}}
\newcommand{\gT}{\mathfrak{T}}
\newcommand{\gU}{\mathfrak{U}}
\newcommand{\gV}{\mathfrak{V}}
\newcommand{\gW}{\mathfrak{W}}
\newcommand{\gX}{\mathfrak{X}}
\newcommand{\gY}{\mathfrak{Y}}
\newcommand{\gZ}{\mathfrak{Z}}

%old calligraphic font
\newcommand{\calA}{\mathcal{A}}
\newcommand{\calB}{\mathcal{B}}
\newcommand{\calC}{\mathcal{C}}
\newcommand{\calD}{\mathcal{D}}
\newcommand{\calE}{\mathcal{E}}
\newcommand{\calF}{\mathcal{F}}
\newcommand{\calG}{\mathcal{G}}
\newcommand{\calH}{\mathcal{H}}
\newcommand{\calI}{\mathcal{I}}
\newcommand{\calJ}{\mathcal{J}}
\newcommand{\calK}{\mathcal{K}}
\newcommand{\calL}{\mathcal{L}}
\newcommand{\calM}{\mathcal{M}}
\newcommand{\calN}{\mathcal{N}}
\newcommand{\calO}{\mathcal{O}}
\newcommand{\calP}{\mathcal{P}}
\newcommand{\calQ}{\mathcal{Q}}
\newcommand{\calR}{\mathcal{R}}
\newcommand{\calS}{\mathcal{S}}
\newcommand{\calT}{\mathcal{T}}
\newcommand{\calU}{\mathcal{U}}
\newcommand{\calV}{\mathcal{V}}
\newcommand{\calW}{\mathcal{W}}
\newcommand{\calX}{\mathcal{X}}
\newcommand{\calY}{\mathcal{Y}}
\newcommand{\calZ}{\mathcal{Z}}


\mode<handout> {
	\usepackage{pgfpages}
	%\setbeameroption{show notes}
	%\pgfpagesuselayout{2 on 1}[a4paper, border shrink=5mm]
	\setbeamercolor{note page}{bg=white}
	\setbeamercolor{note title}{bg=gray!10}
	\setbeamercolor{note date}{fg=gray!10}
}

\usepackage[utf8x]{inputenc}
\usepackage[T2A]{fontenc}
\usepackage[russian]{babel}
\usepackage{tikz}
\usepackage{ragged2e}

\newcounter{saveenumi}
\newcommand{\seti}{\setcounter{saveenumi}{\value{enumi}}}
\newcommand{\conti}{\setcounter{enumi}{\value{saveenumi}}}

\resetcounteronoverlays{saveenumi}

%\usepackage{tikz}
\usetikzlibrary{shapes.geometric, arrows}
\tikzstyle{arrow} = [thick,->,>=stealth]

\setbeamercolor{bluetext_color}{fg=blue}
\newcommand{\bluetext}[1]{{\usebeamercolor[fg]{bluetext_color}#1}}

\newcommand{\bfxi}{\boldsymbol{\xi}}
\newtheorem{definition}{Определение}
\newtheorem{remark}{Замечание}

\title[Теплицев вариант анализа сингулярного спектра]{Теплицев вариант анализа сингулярного спектра}

\author[Потешкин\,Е.\,П., Голяндина\,Н.\,Э.]{\underline{Потешкин Егор Павлович}, Голяндина Нина Эдуардовна}

\institute[Санкт-Петербургский Государственный Университет]{%
	\small
	Санкт-Петербургский государственный университет\\
	Математико-механический факультет\\
	Кафедра статистического моделирования}

%\vspace*{1cm}
\date[Зачет]{\\Наука СПбГУ-2023\\
21 ноября 2023, Санкт-Петербург}

\subject{Talks}	


\begin{document}
	\begin{frame}[plain]
		\titlepage
		\note{Научный руководитель  к.ф.-м.н., доцент Голяндина\,Н.\,Э.,\\
			кафедра статистического моделирования}
	\end{frame}
\begin{frame}{Постановка задачи}
	Временной ряд $\tX=(x_1,\ldots,x_N)$ "--- последовательность наблюдений, упорядоченных по времени.\medskip
	
	\bluetext{Примеры}: Биржевой курс, замеры температуры в течении нескольких лет.\medskip
	
	\bluetext{Дано}: Временной ряд состоит из тренда, сезонности (неслучайные состовляющие) и шума (случайная состовляющая): $\tX=\tT + \tS + \tR$.\medskip
	
	\bluetext{Проблема}: Как выделить неслучайные компоненты?\medskip
	
	\bluetext{Метод}: Singular spectrum analysis (SSA).\medskip
	
	\alert{Задача}: Выделить сигнал как можно точнее.\medskip

    \bluetext{Решение}: Для стационарных рядов использование теплицева варианта SSA. \medskip
    
    \bluetext{Доп.мотивация}: Преимущество теплицева варианта в задача обнаружения сигнала методов Monte Carlo SSA.
\end{frame}
\begin{frame}{Обозначения: оператор вложения и ганкелизации}
	$\tX=(x_1,\ldots,x_N)$. Зафиксируем \emph{длину окна} $L$, $1<L<N$.\medskip
	
	\emph{Оператор вложения} $\cT$:
	\begin{equation*}
		\cT(\tX)=\bfX=\begin{pmatrix}
			x_1 & x_2 & \cdots & x_K \\
			x_2 & x_3 & \cdots & x_{K+1} \\
			\vdots & \vdots & \ddots & \vdots \\
			x_L & x_{L+1} & \cdots & x_N
		\end{pmatrix}
	\end{equation*}
	"--- \emph{траекторная матрица}.\medskip
	
	\emph{Оператор ганкелизации} $\cH$ "--- усреднение матрицы по побочным диагоналям.
\end{frame}
\begin{frame}{SSA: алгоритм}
		\bluetext{Входные данные}: временной ряд $\tX=(x_1,\ldots,x_N)$.\\
	
	\bluetext{Параметр}: длина окна $L$.\\
	
	\bluetext{Результат}: $m$ восстановленных временных рядов.\\
	\begin{figure}
		\scalebox{0.54}{
	\begin{tikzpicture}[node distance=2cm]
		\node[draw, align=center, minimum width=5.5cm, minimum height=2.5cm](1){\textbf{Входные данные}: $\tX$ "--- \\ временной ряд};
		\node[draw, align=center, minimum width=5.5cm, minimum height=2.5cm, below of=1, yshift=-2cm](2){Траекторная матрица\\ $\bf{X}=\cT(\tX)$};
		\node[draw, align=center, minimum width=5.5cm, minimum height=2.5cm, right of=2, xshift=7cm](3){Сумма матриц\\ единичного ранга\\ $\bfX=\sum\limits_{j=1}^d\bfX_j$};
		\node[draw, align=center, minimum width=5.5cm, minimum height=2.5cm, below of=3, yshift=-2cm](4){Сгруппированные\\ матрицы \\ $\bfX=\bfX_{I_1}+\ldots+\bfX_{I_m}$\\$\bfX_I=\sum\limits_{i\in I}\bfX_i$};
		\node[draw, align=center, minimum width=5.5cm, minimum height=2.5cm, below of=2, yshift=-2cm](5){\textbf{Результат}: SSA разложение\\$\tX=\widetilde\tX_1+\ldots+\widetilde\tX_m$\\$\widetilde\tX_{I_k}=\cT^{-1}\circ\cH(\bfX_{I_k})$};
		
		\draw[arrow](1)--node[anchor=west]{1. Вложение}(2);
		\draw[arrow](2)--node[anchor=south]{2. Разложение}(3);
		\draw[arrow](3)--node[anchor=west]{3. Группировка}(4);
		\draw[arrow](4)--node[anchor=south]{4. Восстановление}(5);
	\end{tikzpicture}}
	\caption{Алгоритм SSA}
	\end{figure}
\end{frame}
\begin{frame}{MSSA: алгоритм}
	MSSA "--- обобщение SSA на многомерный случай, когда $\tX$ "--- набор из $D$ временных рядов (каналов). Также этот метод дает преимущество по сравнению с SSA, если сигналы имеют в большой степени одинаковую структуру. \medskip
	
	Пусть $\tX=\{\tX^{(d)}\}_{d=1}^D$ "--- D-канальный временной ряд с длинами $N_1,\ldots,N_D$. Тогда требуется определить только шаг построения траекторной матрицы:\medskip
	
	\begin{itemize}
		\item \emph{Вложение}: составная траекторная матрица $L\times K$  $$\bfX=[\cT(\tX^{(1)}):\ldots:\cT(\tX^{(D)})]=[\bfX^{(1)}:\ldots:\bfX^{(D)}],$$ где $K=\sum_{i=1}^D K_i,$ $K_i=N_i-L+1$.
		
	\end{itemize}
	%\begin{figure}
	%\scalebox{0.65}{
	%	\begin{tikzpicture}[node distance=2cm]
	%		\node[draw, align=center, minimum width=5.5cm, minimum height=2.5cm](1){\textbf{Входные данные}: $\tX$ "---\\ D-канальный временной ряд};
	%		\node[draw, align=center, minimum width=5.5cm, minimum height=2.5cm, below of=1, yshift=-2cm](2){Траекторная матрица \\ $\bf{X}=[\cT(\tX^{(1)}):\ldots:\cT(\tX^{(D)})]$};
	%		\node[draw, align=center, minimum width=5.5cm, minimum height=2.5cm, right of=2, xshift=7cm](3){Сумма матриц\\ единичного ранга\\ $\bfX=\sum\limits_{j=1}^d\bfX_j$};
	%		\node[draw, align=center, minimum width=5.5cm, minimum height=2.5cm, below of=3, yshift=-2cm](4){Сгруппированные\\ матрицы \\ $\bfX=\bfX_{I_1}+\ldots+\bfX_{I_m}$\\$\bfX_I=\sum\limits_{i\in I}\bfX_i$};
	%		\node[draw, align=center, minimum width=5.5cm, minimum height=2.5cm, below of=2, yshift=-2cm](5){\textbf{Результат}: MSSA разложение\\$\tX=\widetilde\tX_1+\ldots+\widetilde\tX_m$\\$\widetilde\tX_{I_k}=\left\{\cT^{-1}\circ\cH\left(\bfX_{I_k}^{(d)}\right)\right\}_{d=1}^D$};
			
	%		\draw[arrow](1)--node[anchor=west]{1. Вложение}(2);
	%		\draw[arrow](2)--node[anchor=south]{2. Разложение}(3);
	%		\draw[arrow](3)--node[anchor=west]{3. Группировка}(4);
	%		\draw[arrow](4)--node[anchor=south]{4. Восстановление}(5);
	%\end{tikzpicture}}
	%\caption{Алгоритм MSSA}
	%\end{figure}
\end{frame}
\begin{frame}{Стационарный случай}

Basic SSA/MSSA --- сингулярное разложение траекторной матрицы, универсальный метод.

\medskip
Toeplitz SSA/MSSA --- теплицево разложение траекторной матрицы, имеет преимущество для стационарных временных рядов. 	
	
\medskip
Остальные этапы кроме разложения не меняются.\medskip
	
%	Тёплицев вариант MSSA был специально разработан для анализа стационарных временных рядов, поскольку имеет преимущество перед стандартным MSSA. \medskip
	
	Два варианта метода Toeplitz MSSA:\medskip
	\begin{enumerate}
		\item Метод Block~[Plaut and Vautard, 1994].\medskip
		\item Метод Sum --- предлагаем.\medskip
	\end{enumerate}
	
\end{frame}
\begin{frame}{Тёплицев MSSA: обозначение}
	Пусть $\tX$ "--- $D$-канальный временной ряд с одинаковыми длинами $N_d=N$ $\forall d=1,\ldots,D$. Зафиксируем $M$.\medskip
	
	Определим матрицу $\bfT_{l,k}^{(M)}\in \mathbb{R}^{M\times M}$ c элементами
	\begin{equation*}\label{eq:block_elements}
	\left(\bfT^{(M)}_{l,k}\right)_{ij}=\frac{1}{N-|i-j|}\sum_{n=1}^{N-|i-j|} x^{(l)}_nx^{(k)}_{n+|i-j|},\ 1\leqslant i,j\leqslant M,
	\end{equation*}
	которая является оценкой ковариационной матрицы $l$ и $k$-го каналов.
\end{frame}
\begin{frame}{Тёплицев MSSA: метод Block}
	\begin{enumerate}
		\item Построить $$\bfT_{\text{Block}}=\begin{pmatrix}
			\bfT^{(K)}_{1,1} & \bfT^{(K)}_{1,2} & \cdots & \bfT^{(K)}_{1,D} \\
			\bfT^{(K)}_{2,1} & \bfT^{(K)}_{2,2} & \cdots & \bfT^{(K)}_{2,D} \\
			\vdots  & \vdots  & \ddots & \vdots  \\
			\bfT^{(K)}_{D,1} & \bfT^{(K)}_{D,D} & \cdots & \bfT^{(K)}_{D,D}
		\end{pmatrix} \in \mathbb{R}^{DK\times DK},$$ где $K = N - L + 1$.\medskip
		\item Найти ортонормированные собственные векторы $Q_1,\ldots,Q_{DK}$ матрицы  $\bfT_{\text{Block}}$ и получить разложение
		\begin{equation*}
			\mathbf{X}=\sum_{i=1}^{DK} (\bfX Q_i) Q_i^\mathrm{T}=\sum_{i=1}^{DK} P_i Q_i^\mathrm{T}.
		\end{equation*}
	\end{enumerate}
\end{frame}
\begin{frame}{Тёплицев MSSA: метод Sum}
	\begin{enumerate}
		\item Построить $\bfT_{\text{Sum}}=\sum_{i=1}^D \bfT^{(L)}_{i,i}\in \mathbb{R}^{L\times L}$.\medskip
		\item Найти ортонормированные собственные векторы $H_1,\ldots,H_L$ матрицы $\bfT_{\text{Sum}}$ и получить разложение
		\begin{equation*}
			\mathbf{X}=\sum_{i=1}^L H_i Z_i^\mathrm{T},
		\end{equation*}
		где $Z_i=\mathbf{X^T}H_i$.
	\end{enumerate}
\end{frame}
\begin{frame}{Численное исследование}
	\bluetext{Дано}: $(\tF^{(1)}, \tF^{(2)})=(\tS^{(1)},\tS^{(2)}) + (\tR^{(1)},\tR^{(2)})$, $N=71$.\medskip
	
	\bluetext{Задача}: проверить точность базового и модифицированных методов для разных значений параметра $L$. \medskip
	
	\bluetext{Рассмотрим 3 случая}:\medskip
	\begin{enumerate}
		\item Косинусы с одинаковыми периодами:
		\[
		s_n^{(1)}=30\cos(2\pi n/12),\quad s_n^{(2)}=20\cos(2\pi n/12),\quad n=1,\ldots, N.
		\]
		\item Косинусы с разными периодами:
		\[
		s_n^{(1)}=30\cos(2\pi n/12),\quad s_n^{(2)}=20\cos(2\pi n/8),\quad n=1,\ldots, N.
		\]
		\item Полиномы первой степени (нестационарные ряды):
		\[
		s_n^{(1)}=1.2n,\quad s_n^{(2)}=0.8n,\quad n=1,\ldots,N.
		\]
	\end{enumerate}

Группировка в SSA: $I_1 = \{1,2\}$.
\end{frame}
\begin{frame}{Численное исследование. Результаты}
		\begin{table}[h]
		\centering
		\caption{MSE восстановления сигнала.}
		\scalebox{0.75}{
		\begin{tabular}{cccccc}\hline
			Случай 1 & $L=12$ & $L=24$ & $L=36$ & $L=48$ & $L=60$\\
			\hline
			SSA & $3.25$ & $\mathbf{2.01}$ & $\mathbf{2.00}$ & $\mathbf{2.01}$ & $3.25$\\
			\hline
			Toeplitz SSA & $3.2$ & $1.87$ & $1.63$ & $\mathbf{1.59}$ & $1.67$ \\
			\hline
			MSSA & $3.18$ & $1.83$ & $1.59$ & $\mathbf{1.47}$ & $2.00$\\
			\hline
			Sum &  $3.17$ & $1.75$ & $1.44$ & $\mathbf{1.32}$ & $\mathbf{1.33}$\\
			\hline
			Block & $1.39$ & $\mathbf{1.26}$ & $\mathbf{1.25}$ & $1.33$ & $1.97$\\
			\hline
		\end{tabular}}
		\scalebox{0.75}{
		\begin{tabular}{cccccc}\hline
			Случай 2 & $L=12$ & $L=24$ & $L=36$ & $L=48$ & $L=60$\\
			\hline
			SSA & $3.25$ & $\mathbf{2.01}$ & $\mathbf{2.00}$ & $\mathbf{2.01}$ & $3.25$\\
			\hline
			Toeplitz SSA & $3.2$ & $1.87$ & $1.63$ & $\mathbf{1.59}$ & $1.67$ \\
			\hline
			MSSA & $6.91$ & $3.77$ & $3.07$ & $\mathbf{2.88}$ & $3.84$\\
			\hline
			Sum & $6.88$ & $3.65$ & $2.64$ & $2.37$ & $\mathbf{2.27}$\\
			\hline
			Block & $4.47$ & $3.67$ & $\mathbf{3.22}$ & $\mathbf{3.23}$ & $3.8$\\
			\hline
		\end{tabular}}
		\scalebox{0.75}{
		\begin{tabular}{cccccc}\hline
			Случай 3 & $L=12$ & $L=24$ & $L=36$ & $L=48$ & $L=60$\\
			\hline
			SSA & $3.65$ & $2.08$ & $\mathbf{1.96}$ & $2.08$ & $3.65$ \\
			\hline
			Toeplitz SSA & $3.33$ & $\mathbf{2.43}$ & $3.74$ & $7.84$ & $16.29$  \\
			\hline
			MSSA & $3.42$ & $1.94$ & $1.63$ & $\mathbf{1.57}$ & $2.27$\\
			\hline
			Sum & $3.32$  & $\mathbf{2.24}$ & $3.04$ & $5.91$ & $11.95$ \\
			\hline
			Block & $12.55$ & $6.18$  & $2.97$ & $\mathbf{1.78}$ & $1.97$\\
			\hline
		\end{tabular}}
		\label{tab:mse}
	\end{table}
\end{frame}
\begin{frame}{Выводы}
	\begin{enumerate}
		\item Sum и Block версии Toeplitz MSSA для стационарного ряда точнее выделяют сигнал, чем Basic MSSA. Если в сигналах разных каналов присутствует одна и та же частота, то Block немного лучше Sum. Но если частоты разные, то Block существенно хуже Sum.\medskip
		\item Рекомендуется использовать длину окна $L\gg(N+1)/2$ для метода Sum и $L\approx (N+1)/2$ для метода Block.\medskip
		\item Если сравнивать по трудоемкости, для оптимальной длины окна метод Sum численно эффективнее Block. Также он позволяет рассматривать многоканальные временные ряды с разными длинами, в отличие от Block.
		\seti
	\end{enumerate}
\end{frame}
%\begin{frame}{Выводы}
%	\begin{enumerate}
%		\conti
%		\item В случае нестационарных рядов (например, в которых присутствует тренд) оба метода показывают результаты хуже, чем Basic MSSA.\medskip
%		\item Применять Block и Sum Toeplitz MSSA рекомендуется к временным рядам, заранее выделив из них тренд.\medskip
%	\end{enumerate}
%\end{frame}
\end{document} 